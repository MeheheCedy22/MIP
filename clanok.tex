% Metódy inžinierskej práce - semestrálny projekt Marek Čederle, AIS ID: 121193

\documentclass[10pt,oneside,slovak,a4paper]{article}

\usepackage[slovak]{babel}
%\usepackage[T1]{fontenc}
\usepackage[IL2]{fontenc} % lepšia sadzba písmena Ľ než v T1
\usepackage[utf8]{inputenc}
\usepackage{graphicx}
\usepackage{url} % príkaz \url na formátovanie URL
\usepackage{hyperref} % odkazy v texte budú aktívne (pri niektorých triedach dokumentov spôsobuje posun textu)

\usepackage{cite}
%\usepackage{times}

\pagestyle{headings}

\title{Vplyv umelej inteligencie na šach\thanks{Semestrálny projekt v predmete Metódy inžinierskej práce, ak. rok 2022/23, vedenie: Ing. Igor Stupavský }}

\author{Marek Čederle\\[2pt]
	{\small Slovenská technická univerzita v Bratislave}\\
	{\small Fakulta informatiky a informačných technológií}\\
	{\small \texttt{xcederlem@stuba.sk}}
	}

\date{\small 6. november 2022}



\begin{document}

\maketitle

\vspace*{\fill}

\begin{abstract}
Ľudstvo sa šachu venuje už stáročia, jeho korene siahajú až do roku 500 pred n. l. Vzhľadom na množstvo ľudí, ktorí ho hrajú, sa v priebehu rokov menili aj jeho pravidlá a formy. Dnes sa šach nehrá len na drevenej šachovnici s ručne vyrobenými figúrkami. S príchodom digitalizácie sa šach dostal na  obrazovky počítačov ako aj superpočítačov. Počas pandémie COVID-19 a po vydaní seriálu Queen‘s Gambit (Dámsky gambit) z produkcie spoločnosti Netflix vypukol šachový boom v online, ale aj v reálnom svete. Po zrušení takzvaných OTB (Over the board) turnajov sa väčšina preniesla do online prostredia. V online prostredí číhajú nástrahy jednoduchého podvádzania pomocou umelej inteligencie (AI), respektíve takzvaných šachových botov. V tomto článku si rozoberieme, prečo a ako umelá inteligencia posúva hranice samotného šachu, ale aj jeho temnejšej strany a to podvádzania. Spomenieme si ako prvý raz umelá inteligencia vyhrala nad človekom a prečo to bol veľmi dôležitý krok ku zlepšeniu šachu ako takého. Na záver sa budeme venovať nedávnej šachovej dráme z turnaja Sinquefield Cup, ktorý sa odohral v St. Louis.
\end{abstract}

\vspace*{\fill}
\pagebreak


\section{Úvod}

Väčšina šachových podujatí, ktoré sa mali konať po začiatku pandémie sa prenieslo do online sveta. Množstvo národných a medzinárodných turnajov sa hralo po prvý raz v histórii online. Počas pandémie sa šach stal jedným z mála športov, ktoré sa udržali. Veľké množstvo šachových veľmajstrov začalo s online prenosmi na platforme Twitch. Šach sa tam stal veľmi populárny a s vysielaním začalo aj veľa iných šachistov. S touto zmenou však  prišli pochybnosti o zručnostiach niektorých hráčov a prvé podozrenia z podvádzania.


\section{Ako funguje umelá inteligencia v šachu}

Keď pozorujeme vývoj umelej inteligencie\footnote{ang. AI - Artificial Intelligence} a šachu, musíme si uvedomiť, ako funguje a aké sú jej korene, aby sme lepšie pochopili možný vplyv umelej inteligencie na budúcnosť šachu.

Umelá inteligencia využíva v superpočítačoch techniky, ktoré im pomáhajú vypočítať ideálny ďalší ťah. Pomocou problému hľadania stromu\footnote{ang. Tree Search Problem}, umelá inteligencia zvažuje aktuálne pozície šachových figúrok na šachovnici. Algoritmus následne udáva ďalší súbor inštrukcií na vykonanie.

Ak by umelá inteligencia používala iba jednoduchý brute force, aby vyskúšala všetky možné pozície (vrátane ilegálnych ťahov), tak už pri 13 figúrkach by sme vyčerpali všetko úložište dostupné na zemi aby sme vôbec tieto dáta uložili, kedže v šachu existuje asi $2^{111}$ až $2^{123}$ možností čo je viacej ako počet atómov v pozorovatelnom vesmíre.

Iný spôsob, ako sa na to pozrieť, je, že umelá inteligencia použije vyššie opísaný algoritmus a identifikuje všetky možné ťahy, ktoré môžu hráči vykonať. Umelá inteligencia má aj funkciu vyhodnocovania, ktorá presne určuje, aký dobrý je daný ťah. Týmto spôsobom pôjde iba do určitej hĺbky a nebude skúšať všetky možné kombinácie.

Vďaka tejto hodnotiacej funkcii môže umelá inteligencia preskúmať akékoľvek usporiadanie hracej plochy a vylúčiť ilegálne ťahy. Tým pádom sa znížia šance súpera na výhru.

Vytvoriť umelú inteligenciu, ktorá má za úlohu hrať šach však nie je jednoduché. Algoritmus stromov je síce zložitý, ale postupne už odborníci prekladajú logické inštrukcie na väčšinu šachových engine-ov aby vytvorili vylepšené neurónové siete.


\section{Pokrok umelej inteligencie vo svete šachu}

Umelá inteligencia prechádza revolúciou, ale zároveň spôsobila revolučné zmeny v celom svete. Šach je jedna z mála oblastí, ktorá inšpiruje pokrok umelej inteligencie. Vývoj umelej inteligencie pre šach pokročil nad rámec hier a zmenil spôsob spolužitia strojov a ľudí. Pokrok umelej inteligencie, zvýraznený strategickými hrami, ovplyvnil mnohé ďalšie oblasti.

Jednou z najznámejších udalostí v boji človeka proti stroju bolo víťazstvo šachového enginu Deep Blue od spoločnosti IBM v roku 1997 proti slávnemu šachovému veľmajstrovi Garrymu Kasparovovi. Toto víťazstvo však bolo dosiahnuté najmä hrubou silou, keďže program Deep Blue prehľadával milióny pozícií za sekundu, aby mohol hrať šachy s o niečo vyššou silou (program vyhral zápas na 6 partií s rozdielom iba jedného bodu), takže asi viac zapôsobilo, že Kasparov s využitím ľudskej intuície dokázal byť stále takmer rovnako silný ako počítač prehľadávajúci milióny pozícií za sekundu. V nasledujúcich rokoch sa výpočtový výkon posunul do takej miery, že ani najlepší šachisti nemajú šancu poraziť moderný šachový engine.\cite{IM20}

Významný pokrok však nastal až o 2 desaťročia neskôr, keď nový šachový engine AlphaZero v roku 2017 vyhral zápas proti FOSS\footnote{ang. Free and Open Source Software - zadarmo, voľne šíriteľný s voľne prístupným zdrojovým kódom} enginu s názvom Stockfish, jednému z najvýkonnejších šachových enginov, aký bol kedy vytvorený, po procese samoučenia z daných dát trvajúcom len 4 hodiny.

Algoritmus AlphaZero sa nesnažil využiť hrubú silu na identifikáciu čo najväčšieho počtu ťahov na šachovnici. Namiesto toho napodobnil proces učenia sa človeka tým, že študoval veľký počtet šachových partií. Tvorcovia AlphaZero tvrdia, že algoritmus sa dokáže naučiť optimalizovať rozhodnutia v akomkoľvek scenári bez zmien alebo usmerňovania, a to bol skutočne prelom.\cite{IM20}

Daľší pokrok nastal, keď 19. novembra 2019 spoločnosť DeepMind predstavila ešte lepší algoritmus založený na spätnoväzobnom učení s názvom MuZero. Tento algoritmus sa naučil hrať šach lepšie ako AlphaZero aj bez toho aby mu niekto na začiatok povedal pravidlá hry.

Učiaci sa algoritmus MuZero má vo vyrovnávacej pamäti uložených maximálne 1 milión šachových partií, pričom paralelne sa hrá 3 000 partií.\cite{IM20} Bolo to možné vďaka tomu, že spoločnosť DeepMind mala prístup k rozsiahlej cloudovej infraštruktúre spoločnosti Google a používala tisíce Tensor\footnote{Novo predstavený čip, ktorý spoločnosť Google používa aj vo svojích mobilných telefónoch aby poháňal funckie spojené s umelou inteligenciou} čipov s procesorovými jednotkami špeciálne navrhnutými na výpočty neurónových sietí.

Neskôr sa aj open-source enginom, ako napríklad Leela Chess Zero, podarilo dosiahnuť úroveň MuZero a dokonca ju prekonať. Bolo to možné vďaka výpočtovému výkonu a vylepšeniam zdrojového kódu, ktoré poskytlo veľké množstvo dobrovoľníkov z celého sveta.


\section{Odhaľovanie podvodov na šachových turnajoch pomocou AI}

Lorem ipsum dolor sit amet


\section{Problémy s detekciou podvodov}

Lorem ipsum dolor sit amet


\section{Nedávna šachová dráma}

Ak sledujete šachový svet, tak ste museli zachytiť nedávnu šachovú drámu spojenú so svetovým šampiónom v šachu, nórom Magnusom Carlsenom. Na turnaji Sinquefield Cup, ktorý sa odohral v St. Louis


\section{Záver}

Lorem ipsum dolor sit amet


\bibliography{literatura}
\bibliographystyle{alpha}
\end{document}